%!TEX root = ../dissertation.tex

\graphicspath{{7-conclusions/figures/}}

\chapter{Conclusions}
\label{ch:conclusions}

By combining deep learning's prodigious capacity to process multimedia data and the practically unlimited source of training data generated by software instruments and data augmentation, this proposal has presented a concrete plan toward a better automatic music transcription system.
Many data-driven methods for music information retrieval have shown that they can perform better than the traditional, heuristic-based methods when provided with enough data for training, and this work will develop further on that, with the help of deep generative models and the huge scale of training data.
These have been only very recently made possible, because of the availability of hardware and software for deep learning at the required scale, as well as the success of the deep generative models especially generative adversarial networks.
This leads to the conclusion that all of the hardware, software, techniques, and the data are pointing to the possibility of deep generative models for automatic music transcription, opening an era for automatic music transcription research.

